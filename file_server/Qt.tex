\documentclass[11pt]{article}

\usepackage[UTF8]{ctex}
\usepackage{amsfonts,amssymb,amsmath}
\usepackage{xcolor}
\usepackage{hyperref}
\usepackage{fancyhdr}
\usepackage{graphicx}
\usepackage{caption}
\usepackage{listings}
\usepackage{hyperref}


\definecolor{mygreen}{rgb}{0,0.6,0}
\definecolor{mygray}{rgb}{0.5,0.5,0.5}
\definecolor{mymauve}{rgb}{0.58,0,0.82}
\lstset{
 backgroundcolor=\color{lightgray}, 
 basicstyle = \footnotesize,       
 breakatwhitespace = false,        
 breaklines = true,                 
 captionpos = b,                    
 commentstyle = \color{mygreen}\bfseries,
 extendedchars = false,             
 frame =shadowbox, 
 framerule=0.5pt,
 keepspaces=true,
 keywordstyle=\color{blue}\bfseries, % keyword style
 language = C++,                     % the language of code
 otherkeywords={string}, 
 numbers=left, 
 numbersep=5pt,
 numberstyle=\tiny\color{mygray},
 rulecolor=\color{black},         
 showspaces=false,  
 showstringspaces=false, 
 showtabs=false,    
 stepnumber=1,         
 stringstyle=\color{mymauve},        % string literal style
 tabsize=2,          
 title=\lstname                      
}

\hypersetup{}

\pagestyle{fancy}
%\fancyhead[C]{\slshape \leftmark}
\fancyhead[R]{note}
\fancyfoot[C]{\thepage}

\allowdisplaybreaks[4]

\graphicspath{{figure/}}



\begin{document}
\tableofcontents

\thispagestyle{empty}
\newpage
\setcounter{page}{1}
\section{Qt部件}
\subsection{connect}
\begin{itemize}

\item 三种响应槽函数的方式, 见
\href{https://doc.qt.io/qt-6/qobject.html#connect}{
\textcolor{blue}{文档}}
\end{itemize}




\subsection{ItemFlags}
\begin{itemize}
\item Qt::NoItemFlags()
\item Qt::ItemIsSelectable()
\item Qt::ItemIsEditable()
\item Qt::ItemIsDragEnable()
\item Qt::ItemIsDropEnable()
\item Qt::ItemIsUserCheckable()
\item Qt::ItemIsEnabled()
\item Qt::ItemNeverHasChildren()
\end{itemize}


\subsection{QString}
\begin{itemize}
\item QString::arg(): 将字符内的\verb|'%N'|转换为arg内的数据.
\end{itemize}




\subsection{tableView}
\begin{itemize}
\item \verb|tableview->selectionModel()->seclectIndexes():|返回一个QModelIndexList对象,实现返回一个选中对象的index.
\item \verb|setItem(row, column, QStandardItem)|:根据行列号更改数据.
\end{itemize}



\subsection{QFile}
\begin{itemize}
\item \href{https://doc.qt.io/qt-6/qudpsocket.html#details}{
\textcolor{blue}{文档}}
\item 定义:\verb|QFile *file = new QFile()|
\item 新建文件:\verb|QFile(filename|
\item \verb|:open(QIODeviceBase::OpenMode mode)|:打开文件,并选择对文件处理的方式.
\item \verb|qint64 QIODevice::write()|::写入文件.
\item \verb||
\end{itemize}

\subsection{QFileDialog}
\begin{itemize}
\item \verb|void QFileDialog::selectFile(const QString &filename)|:选择一个文件.
\item \verb|void QFileDialog::selectNameFilter(const QString &filter)|选择一个filter
\item \verb|getExistingDirectory(QWidget *parent = nullptr, const QString &caption = QString(), const QString &dir = QString(), QFileDialog::Options options = ShowDirsOnly)|
\end{itemize}

\subsection{QDir}
\begin{itemize}
\item \href{https://doc.qt.io/qt-6/qdir.html#homePath}{
\textcolor{blue}{文档}}
\item homepath():返回绝对地址.
\item entryInfoList():返回QFileInfoList对象, 需要选择筛选目录.
\end{itemize}

\newpage



\section{Qt5 布局管理}
\subsection{堆栈窗体}
\begin{verbatim}
QStackedWidget:
connect(list, SIGNAL(currentRowChanged(int)), stack, SLOT(setCurrentIndex(int)));
currentRowChanged:这个信号在用户选择或者改变了部件中当前项的时候被触发。它会发送两个参数,第一个参数是新的当前行的索引,第二个参数是旧的当前行的索引。
setCurrentIndex(int):是一个用于在 Qt 中切换 QStackedWidget 中当前显示页面的方法.
添加正则表达式:
QRegExp regexp("[A-Za-z]");
ui->lineEdit->syValidator(new QRegExpValidator(regExp,this));
QRegExp regexp(筛选的对象, 大小写敏感程度, 筛选方式)

\end{verbatim}
\pagebreak
\subsection{基本布局(了解即可)}
\begin{verbatim}
QFrame:Panel:是一种简单的边框样式,通常用于突出显示或者分隔不同的部分。
QFrame:Raised:凸起。
QFrame:Box:实心的边框。
QFrame:Plain:无阴影。
QFrame:Sunken:它表示一个凹陷的边框效果,通常用于突出显示或者分隔不同的部分。

QGridLayout:
setColumnStretch:是用于设置 QBoxLayout(包括 QHBoxLayout 和 QVBoxLayout)中
子部件列的拉伸因子的方法。
\end{verbatim}

\newpage

\section{模型}
\footnote{u8"中文"可以避免乱码}
\subsection{标准基本对话框}

\begin{verbatim}
QFileDialog://获取用户选择的文件名,保存的文件名,已存在的目录名,文件名列表
getOpenFileName(),getSaveFileName(),getExistingDirectory(),getOpenFileNames
QColorDialog,QFontDialog://获取颜色值,字体
getColor(),getFont()
QInputDialog://标准输入对话框,下拉表条目输入框,int输入框,double类型输入框
getText(),getItem(),getInt(),getDouble()
QMessageBox:
question,information,warning,critical,about,aboutQt
\end{verbatim}
\newpage

\section{TableView}
\begin{itemize}
\item ShowGrid():显示网格.
\item QStandardItemModel:是 Qt 中用于存储结构化数据的模型类之一。它提供了一个表格式的数据结构,可以在其中存储数据.
\item setHorizontalHeaderLabels(model):用于设置表格视图的水平(列)表头标签.
\item setModel(model):将数据模型赋值给表格.
\item setItem(int, int, new StandardItem):添加新数据进表格.
\item setEditTriggers(state):设置可编辑性。
\item QAbstractItemView:提供了一种通用的方法来管理模型中的数据.
\item AllEditTriggers, NoEditTriggers:处于是否可编辑的状态.
\item QModelIndexList: 用于存储QModelIndex对象的容器类.
\item selectionModel()-selectedIndexes():可以获取选中的数据格.
\item 
\begin{verbatim}
T qobject_cast<T>(QObject* object):将父类指针安全地转换为子类指针,
\end{verbatim}
\end{itemize}



\subsection{实现双击tableView触发事件}
\begin{verbatim}
connect(ui->tableView, &QTableView::doubleClicked, this, &MainWindow::openEditPage);
//选择发送对象tableView, 再选择触发事件QTableView::doubleClicked,最后指定触发函数.
\end{verbatim}


\subsection{Model}

\subsubsection{QAbstractItemModel(原型)}
主要用于自定义自己的数据结构
\begin{itemize}
\item 
\href{https://doc.qt.io/qt-5/qabstractitemmodel.html}{
\textcolor{blue}{文档}}
\item 行列相关:RowCount(),InsertRows(),removeRows().
\item \begin{verbatim}
bool QAbstractItemModel::setData(const QModelIndex &index, 
const QVariant &value, int role = Qt::EditRole)
\end{verbatim}
\item \begin{verbatim}
QAbstractItemModel::flags(const QModelIndex &index) const
:返回部件状态.
\end{verbatim}
\item headerData(), headerDataChanged():设置表头,见文档
\item


\end{itemize}



\subsubsection{QStandardItemModel}
\begin{itemize}
\item 
\href{https://doc.qt.io/qt-6/qstandarditemmodel.html#details}{
\textcolor{blue}{文档}}
\item \verb|appendRow(const QList<QStandardItem *> &items)):添加一行数据;|
\item \verb|item(int row, int column = 0):根据行列查询数据|
\item setItem(int row, int column, QStandardItem *item):根据行列设置,设置数据.
\item setRowcount(int rows):将会遗弃比设置行号大的数据.
\item setHorizontalHeaders(int row, QStandardItem *item):设置垂直或者平行表头.
\item clear():清理所有表内数据.
\item data():参考QAbstractItemModel.data():返回一个index指定的数据.
\item 1



\end{itemize}
使用model创建实例:
\begin{lstlisting}
QStandardItemModel model(4, 4);
for (int row = 0; row < model.rowCount(); ++row) {
    for (int column = 0; column < model.columnCount(); ++column) {
        QStandardItem *item = new QStandardItem(QString("row %0, column %1").arg(row).arg(column));
        model.setItem(row, column, item);
    }
}
\end{lstlisting}







\subsubsection{QModelIndex}
\begin{itemize}
\item 存储一个可被QAbstractItemModel使用的index.
\end{itemize}






\subsubsection{QSelectionModel}
\begin{itemize}
\item 
\href{https://doc.qt.io/qt-5/qitemselectionmodel.html#columnIntersectsSelection}{
\textcolor{blue}{文档}}
\item isColumnSelected():
\end{itemize}










\subsubsection{QSortFilterModel}
筛选tableView的流程
\begin{enumerate}
\item 创建QSortFilterModel对象
\item 设置父model
\item 将tableView的model设置为QSortFilterModel对象
\item setFilterKeyColumn(),-1时为全选
\item 设置正则表达式,以及筛选对象等,使用setFilterRegExp.使用QRegExp对象
\end{enumerate}



\newpage
\section{连接数据库}
\subsection{准备工作}
\begin{enumerate}
\item 
\begin{verbatim}lib/libmysql.dll放入msvc2013_64/bin中
\end{verbatim}
\item 
\begin{verbatim}
Src\qtbase\src\plugins\sqldrivers\mysql\mysql.pro修改配置
添加INLUDEPATH:mySQL文件下的INCLUDE目录
添加LIB: 添加lib\libmysql.lib
使用MSVC编译得到的dll文件复制到QT\QT5\5.14.1\msvc2017_64\plugins\sqldrivers
\end{verbatim}
\end{enumerate}



\subsection{建立数据库连接}
\begin{enumerate}
\item 设置数据库类型
\item 设置数据库主机名
\item 设置数据库用户名
\item 设置数据库密码
\item 打开连接
\end{enumerate}


\subsection{访问数据库}



\subsubsection{QSqlQuery}
封装了访问修改执行数据库所需的一些函数.\\
声明:
\begin{lstlisting}
QSqlQuery query("SELECT * FROM DATABASE");
\end{lstlisting}
\begin{itemize}
\item next(): 访问下一个record.
\item prepare(const QString \& query): 存储一条指令,遇到exec()才会执行.
\item \verb|lastError()->TEXT():输出最后一条错误.|
\end{itemize}

\subsubsection{数据库语言}
\begin{verbatim}
新增:
"INSERT INTO table (value1) VALUES (yourvalue)"
查找:
"SECLECT * FROM table WHERE **"
修改:
"UPDATE table SET value1 = yourvalue"
删除:
"DELETE FROM table WHERE **"
\end{verbatim}
\newpage


\section{多线程}
\subsection{QMutex类}
\begin{itemize}
\item mutex.lock():直到调用unlock为止,线程会被阻塞.
\item mutex.tryLock():如果已经被锁定则立即返回.
\end{itemize}

\subsection{QMutexLocker类}
\begin{itemize}
\item 在析构函数中会解锁这个互斥量.
\end{itemize}

\subsection{QThread类}
\begin{itemize}
\item 
\href{https://doc.qt.io/qt-6/qthread.html#details}{
\textcolor{blue}{文档}}
\item started(),finished(): 将在线程创建时发出信号.
\item run():将在创建后进行run()
\item exit(),finish(): 终止线程.
\item wait(): 将线程设置为阻塞态.
\item sleep(): 将线程设置为休眠态.
\end{itemize}


\subsection{QSemaphore}
\begin{itemize}
\item 
\href{https://doc.qt.io/qt-5/qsemaphore.html#details}{
\textcolor{blue}{文档}}
\item acquire():对对象使用P信号.
\item release():对对象使用V信号.
\end{itemize}


\section{Qt网络与通信}
\subsection{QAbstractSocket类}
\href{https://doc.qt.io/qt-6/qabstractsocket.html}{
\textcolor{blue}{文档}}





\subsection{UDP协议传输}
\subsubsection{QUdpSocket类}
\begin{itemize}
\item 
\href{https://doc.qt.io/qt-6/qudpsocket.html#details}{
\textcolor{blue}{文档}}
\item writeDatagram():指定端口和主机号后发送信息.
\item hasPendingDatagrams(): 如果收到至少一条信息, 返回true.
\item readDatagram():读取接收的数据.
\end{itemize}

\subsection{TCP协议传输}


\subsection{QDataStream类}
\begin{itemize}
\item 定义:
\begin{verbatim}
QDataStream::QDataStream(),QDataStream::QDataStream(QIODevice *d)
\end{verbatim}
\item 使用\verb|>>或者<<|将数据传入一个类型内, DataStream内的数据对应减少类型的位数.
\end{itemize}


\subsection{文件传输流程}
传输方:
\begin{enumerate}
\item 建立连接, 使用ByteWritten的方式进行传输.
\item 选择文件后记录文件大小,文件名大小和文件名, 使用二进制数组传入socket.
\item 将剩下的文件使用read(qmin())的形式,分段读取并传送.(使用ByteWritten信号)
\item 在确认传输数据大小与已传输数据大小相等后完成传输.
\end{enumerate}
接收方:
\begin{enumerate}
\item 建立连接, 使得QDateStream连接socket, 接收到二进制数据
\item 首先接收文件名字和文件大小等, 新建一个新的文件.
\item 接接收文件数据本体, 并分段写入文件, 同时记录文件传输进度.(使用readyRead信号)
\item 在确认文件需传送大小与已经传送大小相等时, 记录.\footnote{可能发生粘包, 需要拆包}
\end{enumerate}



\end{document}



